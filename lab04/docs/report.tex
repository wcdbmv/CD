\documentclass{bmstu-gost-7-32}

\begin{document}

\makereporttitle
	{Информатика, искусственный интеллект и системы управления} % Название факультета
	{Программное обеспечение ЭВМ и информационные технологии} % Название кафедры
	{лабораторной работе №~4} % Название работы (в дат. падеже)
	{Конструирование компиляторов} % Название курса (необязательный аргумент)
	{Синтаксический анализатор операторного предшествования} % Тема работы
	{5} % Номер варианта (необязательный аргумент)
	{ИУ7-23М} % Номер группы
	{Керимов~А.~Ш.} % ФИО студента
	{Ступников~А.~А.} % ФИО преподавателя

\section*{Описание задания}

\textbf{Цель работы:} приобретение практических навыков реализации таблично управляемых синтаксических анализаторов на примере анализатора операторного предшествования.

\textbf{Задачи работы:}
\begin{enumerate}
	\item Ознакомиться с основными понятиями и определениями, лежащими в основе синтаксического анализа операторного предшествования.
	\item Изучить алгоритм синтаксического анализа операторного предшествования.
	\item Разработать, тестировать и отладить программу синтаксического анализа в соответствии с предложенным вариантом грамматики.
	\item Включить в программу синтаксического анализ семантические действия для реализации синтаксически управляемого перевода инфиксного выражения в обратную польскую нотацию.
\end{enumerate}

\textbf{Вариант 5. Грамматика G5}

Рассматривается грамматика выражений отношения с правилами

\begin{verbatim}
	<выражение> ->
	    <отношение> { <логическая операция> <отношение> }

	<отношение> ->
	    <простое выражение> [ <операция отношения> <простое выражение> ]

	<простое выражение> ->
	    [ <унарная аддитивная операция> ] <слагаемое> { <бинарная аддитивная операция> <слагаемое> }

	<слагаемое> ->
	    <множитель> { <мультипликативная операция> <множитель> }

	<множитель> ->
	    <первичное> { ** <первичное> } |
	    abs <первичное> |
	    not <первичное>

	<первичное> ->
	    <числовой литерал> |
	    <имя> |
	    ( <выражение> )

	<логическая операция> ->
	    and | or | xor

	<операция отношения> ->
	    < | <= | = | /> | > | >=

	<бинарная аддитивная операция> ->
	    + | - | &

	<унарная аддитивная операция> ->
	    + | -

	<мультипликативная операция> ->
	    * | / | mod | rem

	<операции высшего приоритета> ->
	    ** | abs | not
\end{verbatim}

\textbf{Замечания.}

\begin{enumerate}
	\item Нетерминалы <имя> и <числовой литерал> — это лексические единицы (лексемы), которые оставлены неопределёнными, а при выполнении лабораторной работы можно либо рассматривать их как терминальные символы, либо определить их по своему усмотрению и добавить эти определения.
	\item Терминалы ( ) — это разделители и символы пунктуации.
	\item Терминалы < <= = /> > >= + - * / \& ** - это знаки операций.
	\item Терминалы \textbf{and}, \textbf{or}, \textbf{xor}, \textbf{mod}, \textbf{rem}, \textbf{abs}, \textbf{not} – это знаки операций (зарезервированные).
	\item Нетерминал <выражение> — это начальный символ грамматики.
\end{enumerate}

\section*{Текст программы}

С полным текстом программы можно ознакомиться по адресу: \url{https://github.com/wcdbmv/CD/tree/lab04/lab04/src}.

\section*{Тестирование и результаты}

Выражение:
\begin{verbatim}
	a and c xor ( b or c ) and 1
\end{verbatim}

Результат:
\begin{verbatim}
	a c and b c or xor 1 and
\end{verbatim}

\section*{Выводы}

В данной лабораторной работе были изучены основные понятия и определения, лежащие в основе синтаксического анализа операторного предшествия, изучен алгоритм синтаксического анализа операторного предшествия, разработана, протестирована и отлажена программа синтаксического анализа в соответствии с предложенным вариантом грамматики.
В программу синтаксического анализа были включены семантические действия для реализации синтаксически управляемого перевода инфиксного выражения в обратную польскую нотацию.

\end{document}
